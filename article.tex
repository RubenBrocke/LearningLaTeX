\documentclass{scrartcl}
\usepackage{xcolor}
\usepackage[T1]{fontenc}
\usepackage{listings}
\usepackage{amsmath}

\author{Ruben Brocke}
\title{Programming Language Design}
\subtitle{Creating a programmer friendly language}
\date{\today}

\begin{document}
    \pagecolor{yellow!10}
    \begin{titlepage}
        \clearpage
        \maketitle
        \thispagestyle{empty}
    \end{titlepage}
    \newpage

    \tableofcontents
    \newpage

    \section{Introduction}
    If we look at the first programming languages ever created
    we can see a trend of minimalism, self-service and limited abstraction.
    Throughout the years this has changed. Languages like C++ and Java
    paved the way for more abstraction, garbage collection and faster production.
    Over the years languages have changed from a "machine" friendly syntax
    to a "programmer" friendly syntax and have as a result of this been
    easier to read and faster to prototype.
    \paragraph{}
    The "old" languages are however still widely used. The best example for
    this is the C language that has been thriving for almost 50 years. 
    It boasts amazing performance and little overhead compared to newer languages.
    When programming in C howerver, programmers spend more time translating their
    human thinking to machine thinking. With current year technology we are
    able to create systems that run incredibly fast and can do anything we dream of
    yet we are restricted by our ability to translate out thinking to our machine overlords
    \newpage

    \section{Design}
    \subsection{Classification}
    First of all there needs to be some common way of *ranking* different 
    language syntaxes. This rating should be able to show us how easily 
    the syntax can be used to create an abstract program. There are multiple
    problems when creating such a classification system. These problems mostly
    come in the form of subjectivity. Some programmers feel more at home in
    the machine way of thinking and might even prefer this way. There are however
    many programmers who are frustrated with the lack of abstraction and quick
    prototyping that might allow them to instantly turn their idea's into working code.

    \paragraph{}
    To create this classification we first need to create some classes to use 
    in our classification. These classes should be coherent and easily explainable.
    They need to give insight in the ability of a language syntax to model human abstract thinking.
    To be able quantify human abstract thinking we first need to understand how
    out brains tend to see the world around us. Incidentally we will find out how
    object oriented programming languages gain an edge over their competitors
    \newpage

    \subsection{Purely functional}
    \paragraph{Human abstract thinking}
    Human brains have evolved to understand the world around us. To understand that
    movement of a nearby bush might be a predator, or smoke in the distance
    signaling fire and thus civilization (or a forest fire). In short, Human brains
    work with cause-effect. The better the correlation between this cause and effect
    the more confidence we get in our thoughts. This is one of the beautifully things 
    about computers. They too work with causal cause-effect connections. Nothing in a
    computer does what it was not told to. There are no self moving parts. In other words,
    Computers are deterministic. 

    \subparagraph{}
    It is this deterministic quality that we as human programmers enjoy. It allows
    us to write programs the way we expect them to. It allows is to be able to say that
    when running a program twice, The same thing will happen. However in the current state
    of programming and programming language we do not have confidence in ourselves. 
    We use debuggers and other tools, not to find problems in the computers, but problems in
    ourselves. because in the end, Humans are not deterministic.

    \paragraph{Deterministic vs. non deterministic}
    In the end we find a seeming incompatibility between man and machine.
    We find an impossibility in the way either of the systems work. However
    not all is lost. There is plenty of determinism to be found in humans.
    Most comes in the form of competence. When a human has been able to train
    on something deterministic over and over it is able to start imitating a
    deterministic system. 
    
    \subparagraph{}
    Athletes who train every day on the same task become
    capable of deterministic behavior. They start to show a repetitive pattern
    in their result arriving from the repetitive nature of their craft. This also
    happens in programmers. After multiple years of working the a deterministic 
    system that is programming. They start to show confidence and competence. and
    yet, There it seems to be unable for even the most skilled of programmers (like
    their athlete counterpart) to be deterministic 100\% of the time.

    \paragraph{Forcing determinism in humans}
    To create a programming language suitable for programmers. One quality it might
    have to contains is the ability to ease it's creator into the deterministic mindset.
    A simple way of providing this, is by making sure the creator is unable to add
    seemingly nondeterministic behavior, also known as "side effects". Side effects
    refer to the ability of a program to change it's internal state. The problem happens
    when this internal state is invisible and incomprehensible for the programmer.
    The programmer becomes convinced their program is non deterministic and therefor 
    loses confidence in their craft. 
    
    \subparagraph{}
    Many programming languages are already able to create programs with no
    side effects. These languages are known as purely functional languages. 
    The selling point of purely functional languages are their ability to 
    write programs using only functions. Languages in this catagory are:
    Haskell, Elm and (in a weird way) Lambda calculus. They all have First
    class and higher-order functions and at most a non mutable state.

    \subsection{Object oriented programming}
    \paragraph{Objects and brains}
    Another thing the human brain loves to do is think of everything as an object.
    These objects can be part of a larger object or themselves consist of sub-objects.
    This allows for an easy way of thinking about things around us because in the
    end, everything we see is made out of "something" else. However there are some things
    in our world that cannot be further divided by the "what objects is that object made from"
    operator. For example when we try to disassemble a chair object we might conclude that
    it is made up of legs, a seat and something to rest your back on. When trying to 
    subdivide the legs of our chair however, we are unable to find a meaningful subdivision.
    The best we might find is "wood" or "steel"

    \paragraph{Specifying Objects}
    To make this specific we can use some special syntax to model these correlations.
    This syntax is knows as the Backus–Naur form and is a notation technique for context-free
    grammar. first we will write down our object correlation using normal english and then we
    can try to translate that to Backus–Naur form.
    \newline
    In english:
    \begin{enumerate}
        \item Everything is a collection of objects.
        \item Objects are made up of sub-objects.
        \item Objects can contain non mutable properties.
    \end{enumerate}

    In Backus–Naur form:

    \begin{align*}
         <everything> &::= <object> | <object><everything> \\
         <object> &::= <object><object> \\
         <object> &::= <property><object> \\
    \end{align*}

    \subparagraph{}
    There are a couple of things that need explaining when we look
    at the Backus–Naur form. First of all we need to model the collections
    to make use of recursion. When we (in english) say that everything is 
    collection of objects we see this as a big bucket with objects in it.
    The Backus–Naur form however does not have a way of specifying 
    collections. It fixes this by using recursion. Say everything is
    made up of multiple objects. We can then create an everything containing the 
    first object and another everything. That everything contains the second
    object and so forth for the other objects. This does mess with our english
    understanding of the word "everything" but math doesn't care.

    \paragraph{Other languages}
    A lot of object oriented languages currently available make use of the
    "everything is an object" rule. Most of those however allow some kind of
    mutable state. pretending like a chair can suddenly change the fact that
    it has 4 legs to 6 legs without it begin a new chair with new
    functionality and thus being stateful. 

    \subparagraph{Javascript}
    Javascript is one of those languages that purely sees an object as a
    collection of properties. It does not care what real world object 
    this might be modeling and will happily add whatever to it's objects.
    Not only the values but the properties themselves are mutable.
    \begin{lstlisting}
        chair = { [leg, leg, leg, leg], seat }
        chair.legs = [] // no legs anymore
    \end{lstlisting}

    \subparagraph{C\#}
    C\# is a bit more restrictive with what it defines as an object. Generally
    these objects are classes which the programmer can create. Classes have a
    predefined set of properties and functionality who's value is mutable.
    \begin{lstlisting}
        class Chair
        {
            public Leg[] legs
            public Seat seat
        }
    \end{lstlisting}
    
    \subparagraph{Haskell $\lambda$}
    Haskell has something else again. Instead of classes or objects it uses
    typeclasses. Don't let the name fool you because they're quite different from
    the C\# classes. A typeclass can be seen as a sort of interface. It models
    behavior and not properties.
    \begin{lstlisting}
        data Chair = Chair Leg Leg Leg Leg Seat
    \end{lstlisting}

    \paragraph{Human friendly object orienting}
    In a new language we would look for an easy to understand syntax that is
    able to model what we see around us. It should for example be able to
    say that a chair always has a collection of legs. The amount of legs
    defines the chair. Any change in legs would lead to the creation of a new
    chair and with that be a new object. This should be obvious from syntax.

    \subsection{Words > symbols}
    \paragraph{Translating}
    When writing a program we humans prefer to be able to read it as if it were
    english. If it is not we will frequently think of some translation that makes
    it easier to understand. For example the following statement: $x = 4$ we
    translate to "x is equal to 4". Of course it would be counter-productive for
    any language to expect the programmer to write "is equal to" every time they
    set the value of a variable. For these easy to learn translation we prefer
    symbols. There are some statements in programming however that we don't like
    to translate. A good example of this is the $foreach$ operator found in many
    languages. Math does have some way of writing this down in symbols. this being:
    $\forall x \in A$ which could be translated to "for all x in A" which looks
    like our foreach. This is however a hassle to write down and does not intuitively
    show what is happening. To give a concrete example we can write some function
    that doubles every number from 0 to 100.
        \begin{lstlisting}
            foreach n in 0..100
                n = n * 2
        \end{lstlisting}
    We can also write this in the symbol way like this:
    \begin{equation*}
        \forall x \in \left\{ 0,...,100 \right\} x = x * 2
    \end{equation*}
    In terms of readability (maybe not elegance) the first example would be better.

    \paragraph{Code Blocks}
    Another construct that is often used in object oriented languages is the use
    of code blocks. These often use the \{ and \} as limiters (most C-like languages)
    but can also use "do" and "end" (in the case of ruby) or tab indentation (python).
    If we try our translation technique on code blocks for an if statement we often
    get the "do" and "end" like ruby. for example:
    \begin{lstlisting}
        if (something)
        {    
            function1()  
        }
        else
        {
            function2()
        }
    \end{lstlisting}
    We would translate this statement to "if something then do function1 else
    do function2 and then end". containing the words "then", "do" and "end". 

    \subsection{Inferred static typing}
    Lot's of programming languages make a difference between "5" and "five".
    This is because $"five" + "five" \neq "ten"$ but $5 + 5 = 5$. In other
    words some data has different functionality from others. Some language choose
    to make this completely unambiguous by forcing you to specify what type
    of data it is. Other languages allow for conversion between data types.
    which will effectively make $"5" == 5$. 

    \paragraph{Static typing}
    Static typing is the act of not changing the type of a variable. The type
    needs to be clear before the program is run and cannot be changed. Statically
    typed languages do allow the creation of a different data type using another.
    For example we can do $5.toString()$ and that will give us the string "5".

    \paragraph{Type inference}
    Another technique used by some language is type inference. This is the act
    of assuming data types based on their values. For example if we do $x = 5$
    we can infer x is of type Integer or Number. When we however do $x = "5"$
    we can infer x is of type String. Lastly let us consider something like this:
    \begin{lstlisting}
        variable = new Chair()
    \end{lstlisting}
    We can infer that variable is now of type chair. When we later try this:
    $variable = 5$ we can tell the user that we know "variable" is of type Chair
    and 5 is not of type Chair.

    \subsection{Summary}
    To summarize there are 3 qualities we want our human friendly programming
    language to have. these are:
    \begin{enumerate}
        \item Purely functional to avoid side effects
        \item Object oriented to model the real world
        \item Words over Symbols to create readability
        \item Inferred static typing for ease and safety
    \end{enumerate}
\end{document}